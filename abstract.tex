\documentclass[12pt,a4paper]{article}
\usepackage{graphicx}
\usepackage{hyperref}
\begin{document}
\begin{titlepage}
	\centering
	\includegraphics[width=1.2\textwidth]{src/logo-jkuat.jpg}\par\vspace{1cm}
	\vspace{1cm}
	{\scshape\Large Department of Telecommunication and Information Engineering\par}
	\vspace{1cm}
	{\scshape\Large Unit: Project\par}
	\vspace{1cm}
	{\scshape\Large Year of Study: 5\par}
	\vspace{1cm}
	%{\scshape\Large Unit Code: ETI 2506\par}
%\vspace{1cm}
	{\scshape\Large CONCEPT PAPER\par}
	\vspace{1cm}
	
	
	{\Large\itshape Clifford Beta \par}
	{\Large\itshape EN 273-0628/2013\par}
	{\Large\itshape Anne Okemwa \par}
	{\Large\itshape EN 273-0628/2014\par}
	\vfill




% Bottom of the page
	{\large \today\par}

\end{titlepage}
\newpage
Millimetre Waves (mmWave) systems have the potential of enabling multi-gigabit-per-second communications with very low latency in future Intelligent Transportation Systems (ITSs). Typical application areas would be, autonomous driving, immersive gaming, Augmented Reality and Virtual Reality.

Due to increased vehicular mobility, there is need of frequent antenna beam realignments, thereby significantly increasing the in-Band beam forming overhead.

This project aims to develop an Adaptive antenna beamforming algorithm using a recurrent neural network approach. The RNN algorithm of choice in this case is the Long Short Term Memory Neural Network trained on vehicle trajectory dataset to predict vehicular motion.
Based on the prediction of the LSTM, we will steer the antenna beam in anticipation of the motion.

We hope to demonstrate that the suggested approach is more efficient that the current approaches to beam steering and has the potential of increasing the data rates and the quality of service (QoS). This will be possible because vehicular motion can be modelled as a sequence, and LSTMs perform well in sequence prediction and generation problems. 



\end{document}


